\documentclass{article}

\usepackage{helvet}
\usepackage[a4paper,margin=1in]{geometry}

\renewcommand{\familydefault}{\sfdefault}
\author{Amartya Vadlamani}
\title{Using Machine Learning and Natural Language Processing to Automatically Diagnose Medical Conditions}
\date{}
\begin{document}
\maketitle
\pagenumbering{gobble}

\paragraph {}
With an growing and aging population, getting fast access to medical advice is
vital to ensuring a good quality of life but ever growing pressures on the NHS
mean that it can take between 3 and 18 weeks to see a
specialist\cite{referaltime}, and up to four hours to go into a walk-in clinic
to see a nurse if you ever get seen at all\cite{appointmentwait}. Online
symptom checkers aren't that useful either as according to a BMJ study there is
only a 57\% chance that correct diagnosis will be in the top 20
options\cite{webMD}. The bottleneck in all of these cases is the lack of
doctors who can interview patients and produce a diagnosis quickly and reliably.

\paragraph{}
Now if you break up the tasks doctors do while in a diagnostic appointment;
process human input; analysis that input to find relations and patterns; and
produce a given treatment plan; the task seems ripe for automation by
computers. There are two fields in Computer Science, Machine Learning and
Natural Language Processing, that started in the 50s but only recently had
enough computing power to become applicable to many different tasks. Machine
Learning lets machines learn from structured datasets, i.e.  datasets in a
machine readable format, and perform tasks that they could never be manually
programmed to do.\cite{whatiswatson} While this is all well and good, where
would you find a structured dataset large enough to teach a computer the whole
of medicine? This is where Natural Language Processing comes in. This subfield
of computational linguistics is based around finding methods that computers can
turn the unstructured datasets that people produce when they write and into
structured datasets that a computer can understand.\cite{whatiswatson} Put
these two fields together and feed it from a medical database of thousands of
research papers and you get a digital doctor that scales with the number of
computers that you can attach it to rather than the number of people that have
graduated from medical school in the last few years and can always be
constantly learning and applying the latest research in ways that humans just
cannot. That`s the theory at least.

\paragraph{}
In practice, there are a few teething issues but the technology is now mostly
here, in the form of machine learning systems like IBM's
Watson\cite{whatiswatson}. A question answering computer program that is able
to answer questions in plain English. A point to note is that AI systems like
Watson do not have to be perfect just better than people and in some instances
they already are. Watson, for example, already diagnoses lung cancer with 90\%
accuracy as opposed to 50\% accuracy by human doctors and has made a leukemia
diagnosis that had stumped a panel of human doctors for months.
\cite{watsonbeatshumans}\cite{iswiredavalidcite}. 

\paragraph{}
Now while these figures show quite a lot of promise there still a long way
before a future of fast and easy access to digital doctors. But the technology
is currently expensive as the algorithms need a lot of computation power to run
and this means that large data centers need to either be set up or rented and
while strides are being made to make these systems more efficient they still
need large computer arrays to serve the ever growing population that was the
catalyst for their creation. As well as this more data needs to be gathered
about where the inherent biases are in this system as well as reducing the
number of times that it makes false predictions all of this raising concerns
about data security and privacy. But all this could mean the earlier diagnosis and
treatment of a whole range of time sensitive diseases like cancer leading to
earlier treatment and higher survival rates.

\begin{thebibliography}{6}
    \bibitem{referaltime} 
        NHS UK. Guide to NHS waiting times in England .
        http://www.nhs.uk/NHSEngland/appointment-booking/Pages/nhs-waiting-times.aspx
        (accessed 05 December 2016).
    \bibitem{appointmentwait}
        Mark Tran. GP appointment waiting times crisis revealed in Labour
        research.
        https://www.theguardian.com/society/2016/apr/17/gp-appointment-waiting-times-crisis-revealed-in-labour-research
        (accessed 05 December 2016).
    \bibitem{clinicwaittime}
        Central London Community Healthcare. NHS Walk-in Centres.
        http://www.clch.nhs.uk/services/walk-in-and-urgent-care-centres.aspx
        (accessed 05 December 2016).
    \bibitem{webMD}
        Hannah L Semigran, Jeffrey A Linder, Courtney Gidengil, Ateev Mehrotra.
        Evaluation of symptom checkers for self diagnosis and triage: audit
        study. BMJ (British Medical Journal) 2015;
        http://www.bmj.com/content/351/bmj.h3480 (accessed 05 December 2016).
    \bibitem{whatiswatson}
        David Ferrucci ⁎, ferrucci@us.ibm.com, Anthony Levas, Sugato Bagchi,
        David Gondek, Erik T. Mueller. Watson: Beyond Jeopardy!. Science Direct
        2012;
        http://www.sciencedirect.com/science/article/pii/S0004370212000872
        (accessed 05 December 2016).
    \bibitem{watsonbeatshumans}
        Alfred Ng. IBM’s Watson gives proper diagnosis for Japanese leukemia
        patient after doctors were stumped for months .
        http://www.nydailynews.com/news/world/ibm-watson-proper-diagnosis-doctors-stumped-article-1.2741857
        (accessed 05 December 2016).
    \bibitem{iswiredavalidcite}
        Ian Steadman. IBM's Watson is better at diagnosing cancer than human
        doctors. http://www.wired.co.uk/article/ibm-watson-medical-doctor
        (accessed 05 December 2016).
\end{thebibliography}

\end{document}
